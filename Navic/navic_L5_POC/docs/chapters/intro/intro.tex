%\documentclass{article}

%\usepackage{amssymb, amsfonts,amsthm,amsmath}
%\usepackage{enumitem}
%\usepackage{hyperref,xcolor}

%\def\inputGnumericTable{}
%\usepackage{array}
%\usepackage{longtable}
%\usepackage{calc}
%\usepackage{multirow}
%\usepackage{hhline}
%\usepackage{ifthen}



%\begin{document}
%\title{Details of the NavIC frequency bands }
%\author{\Large Shreyash Putta - FWC22070}
%\date{}

%\maketitle

NavIC (an acronym for 'Navigation with Indian Constellation') is the operational name for Indian Regional Navigation Satellite System (IRNSS), developed independently and indigenously by Indian Space Research Organization (ISRO). The objective of this autonomous regional satellite navigation system is to provide accurate real-time positioning and timing services to users in India and a region extending upto $1,500$ km ($930$ mi) around it. 
\\
\\
NavIC is designed with a constellation of $7$ satellites and a network of ground stations operating $24$ x $7$. Three satellites of the constellation
are placed in geostationary orbit and four satellites are placed in inclined geosynchronous orbit. The ground network consists of control centre, precise timing facility, range and integrity monitoring stations, two-way ranging stations, etc.
\\
\\
NavIC provides two levels of service, the "standard positioning service", which is open for civilian use, and a "restricted service" (an encrypted one) for authorised users (including the military). NavIC has a theoritical positional accuracy of $5$m - $20$m for general users and $0.5$m for military purposes.
\\
\\
This book describes the Real time NavIC standards simulation using C code. Chapter 2 provides the information about the NavIC transmitter and how it is implemented in real time. Chapter 3 describes how the NavIC receiver is implemented in realtime and Chapter 4 provides the requirements and complete real time implementation of NavIC Transmitter and Receiver and Chapter 6 details out key
results from the simulation. 
\\
\\
\section{Scope of simulation}	
The scope of the simulation is limited to 
\begin{enumerate}
	\item Generating the Real time NavIC Navigation data corresponds to the receiver location that contain the information of position of satellites in the orbit.
	\item Generating the NavIC baseband samples from the navigation data.
	\item Transmit the basband samples by mixing with carrier signal with L5 frequency to air.  
	\item Receive the L5 signals from air and down convert to baseband and feed it to the receiver module for computing the location of the receiver.
	\item only SPS services signal (RS signal is out of scope) 
\end{enumerate}



%\end{document}
